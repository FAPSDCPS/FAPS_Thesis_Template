
\usepackage[english,german]{babel} 	% Sprachpaket
% \usepackage[latin1]{inputenc} 		% Konvertierungspaket (Sprachen:Westeuropa)
\usepackage[utf8]{inputenc} 		% joba Konvertierungspaket (Sprachen:Westeuropa)

\usepackage[centertags]{amsmath}		% Mathepakete
\usepackage{amssymb}
\usepackage{array}

\usepackage{eso-pic}
\usepackage[pdftex]{graphicx}
\usepackage{epstopdf}

\usepackage[absolute]{textpos}

\usepackage[format=hang, font=footnotesize, labelfont=bf]{caption}  % Layout fuer Bildbeschriftung

\usepackage{fancyhdr}	% Definition von Kopf- und Fusszeilen

\usepackage{color}		% fuer farbige Texte

\usepackage{calc}

%\usepackage{tocstyle}	% Package fuer Layout des Inhaltsverzeichnisses
\usepackage{tocloft}	% Package zum Bearbeiten des Layouts des Abbildungs- und Tabellenverzeichnisses
\usepackage{titletoc}	% Package zum Bearbeiten des Layouts des Inhaltsverzeichnisses

\normalsize

%-----------------------------------------------------------------------------------------------------------%

\usepackage{textcomp}

\usepackage{subcaption}	% Subcaptions

\usepackage{multirow}	% Multirow - Tabellen
\usepackage{longtable}	% Lange Tabelle

\usepackage{color}		% fuer Farben im allgemeinen
\usepackage{colortbl}	% fuer die Hintergrundfarbe einzelner Zellen in Tabellen

\usepackage{url}		% URL-Package

\usepackage{ifthen}		% if then
\usepackage{nomencl}	% Abkuerzungen und Verzeichnisse
\usepackage{makeidx}	% Index erstellen

\usepackage{blindtext}	% Lorem ipsum

\usepackage[]{hyperref}	% Links usw.

\usepackage{setspace}	% Zeilenabstand

\usepackage[bottom]{footmisc}	% Fussnotenposition
\usepackage{chngcntr}

\usepackage{pdfpages}

\ifthenelse{\equal{\Schriftart}{serifenlos}}{
	\usepackage[scaled=0.92]{helvet}
	\renewcommand{\familydefault}{phv}
}{}


