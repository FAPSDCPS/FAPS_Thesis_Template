% Pakete
\usepackage[ansinew]{inputenc}   % UTF8-Kodierung f�r Umlaute
\usepackage[T1]{fontenc}      % use TeX encoding then Type 1.
\usepackage{lmodern}          % Bessere Schriftart (ersetzt CM-Schriften)
\usepackage{ngerman}          % deutsche Silbentrennung
\usepackage{graphicx}         % Einbindung von Grafiken
\usepackage{fancyhdr}         % eigenes Layout einbinden
\usepackage{pdfpages}         % PDF-Seiten einbinden
\usepackage{xifthen}          % Wenn-dann-Abfragen
\usepackage{charter}          % Charter-Schrift
\usepackage{titlesec}         % Anpassung der �berschriften
\usepackage{longtable}        % Tabellen �ber Seitenumbruch hinweg
\usepackage{setspace}         % Zeilenabstand festlegen
\usepackage{hyperref}         % Hyperlinks und interne PDF-Verweise

% Layout
\hbadness=10000                % unterdr�ckt unwichtige Fehlermeldungen
\clubpenalty = 300 %10000           % Keine "Schusterjungen"
\widowpenalty = 300 %10000          % Keine "Hurenkinder"
\displaywidowpenalty = 300 %10000

% Linie im Kopf ausblenden und im Fu� eine feine Linie
\renewcommand*{\headrulewidth}{0pt}
\renewcommand*{\footrulewidth}{0.4pt}

% �berschriftenlayout ver�ndern
\titlespacing{\section}{0mm}{2em}{2em}
\titlespacing{\subsection}{0em}{0em}{0em}
\titleformat{\section}{\normalfont\LARGE\scshape}{}{0mm}{\hspace*{\fill}}
\titleformat{\subsection}{\normalfont\Large\bfseries\scshape}{}{0mm}{}[\vskip-1.5em\hrulefill]

% Eigene Fusszeile festlegen, Kopfzeile leeren
\pagestyle{fancy}
\fancyhead{}
\cfoot{\VollerName, \AbsenderStrasse, \AbsenderPLZOrt, \Telefon}

% Tabelle f�r Lebenslauf
\newlength{\AbstandAbschnitt}
\newboolean{UnterabschnittBegonnen}
\newenvironment{Abschnitt}[1][0em]
{%
    \setlength{\AbstandAbschnitt}{1cm}%
    \begin{longtable}{p{0.2\linewidth}p{0.07\linewidth}p{0.65\linewidth}}%
    \setboolean{UnterabschnittBegonnen}{false}%
}
{%
    \end{longtable}%
    \vspace{\AbstandAbschnitt}%
}

% Unterabschnitt in Lebenslauftabelle mit sonstigen Themen
% Achtung, keine Umgebung, sondern ein Kommando!
\newlength{\AbstandUnterabschnitt}
\newcommand{\Unterabschnitt}[3][0em]
{%
    \end{longtable}%
    \ifthenelse{\boolean{UnterabschnittBegonnen}}{%
        \setlength{\AbstandUnterabschnitt}{-2em}%{-5em}%
    }{%
        \setlength{\AbstandUnterabschnitt}{-2em}%
    }%
    \addtolength{\AbstandUnterabschnitt}{#1}%
    \vspace{\AbstandUnterabschnitt}%
    \textit{#2} #3%
    \setboolean{UnterabschnittBegonnen}{true}%
    \begin{longtable}{p{0.3\linewidth}p{0.64\linewidth}}%
    \\[1em]
}



